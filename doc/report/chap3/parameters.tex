\chapter{参数选择及收敛条件}
\label{chp:3}
我们进行 M 次模特卡罗模拟实验,对于每一次模拟,选取某个 $N$, 计算相应的误差$M_T^N$, 我们希望该误差小于可接受的最大误差 $\epsilon=X0*10^{-3}$。 
我们用蒙特卡洛的方法计算 $M_T^N\leq \epsilon$ 的概率,并希望该概率能超过某一个特定值: $Prob=95\%$。 
在这个问题中 $M$ 和 $N$ 的初始值以及其更新方法至关重要。 对$M$ 而言, 一方面较大的 $M$ 会增大模特卡罗方法以及通过此方法计算出来的概率的可信度,另一方面
$M$ 过大,将不可避免的造成更高的资源和硬件的要求,如何在保持一定可行度的条件下选择一个较小的 $M$ 需要一个精巧的方法。
对 $N$ 而言,较大的 $N$ 会使得逼近更准确,从而误差 $M_T^N$ 较小,但也会造成资源的浪费和迭代收敛的过程冗长。
较大或者较小的 $M$ 和 $N$ 的初始值将导致收敛缓慢甚至不收敛。
如何选择合适的 $M$ 和 $N$ 的初始值,
如何让它们互相迭代更新, 以及怎样判断收敛与否是本章所需要解决的问题。
\section{参数 N} % (fold)
\label{sec:N}
关于 $N$ 的初始值的选取,一个首要的问题是,数值模拟中,$M$ 会影响我们对 $N$ 的判断,甚至一个选取不好的 $M$ 会给出错误的 $N$ 的更新方向,从而导致问题
得不到收敛。为了避免这种问题, 我们在这里给出一种不依赖于 $M$ 的 $N$ 的初始值选定方法。

我们需要一个尽可能小的 $N$, 使得存在某个关于 $N$ 的函数 $P(N)$,如下不等式成立:
\begin{equation}
Probability(M_T^N\geq\epsilon)=\mathbb{P}(M_T^N\geq\epsilon)<P(N)。
\end{equation}
若我们能发现这样的函数$P(N)$,则所有满足不等式:
\begin{equation}
P(N)\leq 1-Prob=5\% 
\end{equation}
的正整数 $N$, 都满足 $\mathbb{P}(M_T^N<\epsilon)>Prob=95\%$。
我们只需要取满足不等式的最小的正整数 $N$ 即可。此时 $N$ 的选取独立于 $M$, 也即无论我们选取怎样的$M$, 理论上都会有 $\mathbb{P}(M_T^N<\epsilon)>Prob$ 成立。

为了寻找函数 $P(N)$, 我们先引入两个引理,

\textbf{(Burkholder-Davis-Gundy inequality)}
Let $T>0$ and $(M_t)_{0\leq t\leq T}$ be a continuous local martingale such that $M_0=0$. For every $0<p<+\infty$, there exist universal constants $c_p$ and $C_p$, independent of $T$ and $(M_t)_{0\leq t\leq T}$ such that 
\begin{equation}
c_p\mathbb{E}[\left \langle M \right \rangle_T^{\frac{p}{2}}]\leq \mathbb{E}[(sup_{0\leq t\leq T}|M_t|)^p]\leq C_p\mathbb{E}[\left \langle M \right \rangle_T^{\frac{p}{2}}]
\end{equation}

See XXXXXX for a proof.

In paper XXXXXX, The authors proved $C_p\leq (2\sqrt{p})^p$.

\textbf{(Generalized Minkowski inequality)}
Suppose that $(S_1,\mu_1)$ and $(S_2,\mu_2)$ are two measure spaces and $F : S_1\times S_2\rightarrow R$ is measurable. Then generalized Minkowski's integral inequality is :
\begin{equation}
(\int_{S_2}|\int_{S_1}F(x,y)d\mu_1(x)|^pd\mu_2(y))^{\frac{1}{p}}\leq \int_{S_1}(\int_{S_2}|F(x,y)|^pd\mu_2(y))^{\frac{1}{p}}d\mu_1(x), \forall p\geq 1
\end{equation}

See XXXXXX for a proof

With the notations as before and  assume that the interest rate $r=0$, volatility $\sigma(X_t)=constant=\sigma>0$ and payoff function $f(x)=(x-K)^{+}, K>0$. Then 
\[
\mathbb{P}(M_T^n\geq \gamma)\leq 
\begin{cases}
\frac{128\sigma^{\frac{1}{2}}x_0^2T^{\frac{5}{4}}e^{\sigma^2}}{e^{\frac{1}{4}}\gamma^2\pi (\log(\frac{x_0}{K})+\frac{1}{2}\sigma^2T)^{\frac{1}{2}}(2\pi)^{\frac{1}{4}}}\frac{1}{n}, \quad \log(C_1n)<3\log(2)+3\\
T^{\frac{1}{4}}(\frac{\sigma}{(\log(\frac{x_0}{K})+\frac{1}{2}\sigma^2T)\sqrt{2\pi}})^{\frac{1}{2}}e^{-\frac{1}{4}}e^{-\frac{3\gamma^{\frac{2}{3}}\pi^{\frac{1}{3}}}{2ex_0^{\frac{2}{3}}T^{\frac{1}{3}}16^{\frac{1}{3}}e^{\frac{\sigma^2}{3}}}n^{\frac{1}{3}}}, \quad \log(C_1n)\geq 3\log(2)+3\\
\end{cases}
\]

% section 参数N (end)

\section{参数M} % (fold)
\label{sec:M}
在确定了 $N$ 的迭代初始值之后 $N_0$,我们可以使用该值来确定一个合适大小的 $M$ 的迭代初始值。
一个合适的 $M$ 的初始值应该满足如下条件:在 $N$ 为其初始迭代值时, $M$ 的值应该使得模特卡罗模拟出的概率 $\mathbb{P}(M_T^N<\epsilon)$ 收敛稳定。
也即,固定$N$ 为其初始值不变,当 $M$ 在其迭代初始值附近扰动时,相应的 $\mathbb{P}(M_T^N<\epsilon)$ 也只是微小扰动。动态调优的技术可以被应用在这里。

在我们的问题背景下,通过动态调优和经验发现,一个较为恰当的 $M$ 初始值可以为 $M=10^6$, 我们通过比较 $M=10^6+i*100$, $i=0,...,10$ 
所对应的 $\mathbb{P}(M_T^N<\epsilon)$来调整 $M$, 若对应的概率在 $10^{-3}$范围内,则将$M$ 减小为原先的一半,否则增大 $M$ 为原先的两倍。
重复此过程,知道发现满足条件的最小的 $M$。 我们然后固定这个最优的 $M$, 然后尝试找出最优的 $N$。

根据前面的理论,最优的 $N$ 存在于 $[0, N_0]$ 区间内。我们仍然使用二分法寻找最优的$N$, 首先我们令 $N_1=N_0/2$, 在此参数下模拟相应的
$\mathbb{P}(M_T^N<\epsilon)$, 若此模拟概率大于 $Prob$, 则令 $N_2=N_1/2$, 反之,则令 $N_2=\frac{N_0+N_1}{2}$。
重复此过程直至找到最优的$N$。

% section 参数M (end)




