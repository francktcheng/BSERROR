\chapter{参数选择及收敛条件}
\label{chp:3}
我们进行 M 次模特卡罗模拟实验,对于每一次模拟,选取某个 $N$, 计算相应的误差$M_T^N$, 我们希望该误差小于可接受的最大误差 $\epsilon=X0*10^{-3}$。 
我们用蒙特卡洛的方法计算 $M_T^N\leq \epsilon$ 的概率,并希望该概率能超过某一个特定值: $Prob=95\%$。 
在这个问题中 $M$ 和 $N$ 的初始值以及其更新方法至关重要。 对$M$ 而言, 一方面较大的 $M$ 会增大模特卡罗方法以及通过此方法计算出来的概率的可信度,另一方面
$M$ 过大,将不可避免的造成更高的资源和硬件的要求,如何在保持一定可行度的条件下选择一个较小的 $M$ 需要一个精巧的方法。
对 $N$ 而言,较大的 $N$ 会使得逼近更准确,从而误差 $M_T^N$ 较小,但也会造成资源的浪费和迭代收敛的过程冗长。
较大或者较小的 $M$ 和 $N$ 的初始值将导致收敛缓慢甚至不收敛。
如何选择合适的 $M$ 和 $N$ 的初始值,
如何让它们互相迭代更新, 以及怎样判断收敛与否是本章所需要解决的问题。
\section{参数 N} % (fold)
\label{sec:N}
关于 $N$ 的初始值的选取,一个首要的问题是,数值模拟中,$M$ 会影响我们对 $N$ 的判断,甚至一个选取不好的 $M$ 会给出错误的 $N$ 的更新方向,从而导致问题
得不到收敛。为了避免这种问题, 我们在这里给出一种不依赖于 $M$ 的 $N$ 的初始值选定方法。

我们需要一个尽可能小的 $N$, 使得存在某个关于 $N$ 的函数 $P(N)$,如下不等式成立:
\begin{equation}
Probability(M_T^N\geq\epsilon)=\mathbb{P}(M_T^N\geq\epsilon)<P(N)。
\end{equation}
若我们能发现这样的函数$P(N)$,则所有满足不等式:
\begin{equation}
P(N)\leq 1-Prob=5\% 
\end{equation}
的正整数 $N$, 都满足 $\mathbb{P}(M_T^N<\epsilon)>Prob=95\%$。
我们只需要取满足不等式的最小的正整数 $N$ 即可。此时 $N$ 的选取独立于 $M$, 也即无论我们选取怎样的$M$, 理论上都会有 $\mathbb{P}(M_T^N<\epsilon)>Prob$ 成立。

为了寻找函数 $P(N)$, 我们先引入两个引理,

\textbf{(Burkholder-Davis-Gundy inequality)}
令 $T>0$,$(M_t)_{0\leq t\leq T}$ 为一个局部的鞅,且 $M_0=0$. 则对于所有的 $0<p<+\infty$, 存在独立于 $T$ 和 $(M_t)_{0\leq t\leq T}$ 的常数
$c_p$,$C_p$, 使得 
\begin{equation}
c_p\mathbb{E}[\left \langle M \right \rangle_T^{\frac{p}{2}}]\leq \mathbb{E}[(sup_{0\leq t\leq T}|M_t|)^p]\leq C_p\mathbb{E}[\left \langle M \right \rangle_T^{\frac{p}{2}}]
\end{equation}
证明:见\cite{MouvementBrownien}.

在文章中\cite{LPEstimates}中,作者证明了 $C_p\leq (2\sqrt{p})^p$.

\textbf{(Generalized Minkowski inequality)}
设 $(S_1,\mu_1)$ 和 $(S_2,\mu_2)$ 是两个可测空间,且 $F : S_1\times S_2\rightarrow R$ 为可测映射. 
则广义的Minkowski 积分不等式是:
\begin{equation}
(\int_{S_2}|\int_{S_1}F(x,y)d\mu_1(x)|^pd\mu_2(y))^{\frac{1}{p}}\leq \int_{S_1}(\int_{S_2}|F(x,y)|^pd\mu_2(y))^{\frac{1}{p}}d\mu_1(x), \forall p\geq 1
\end{equation}
证明:见\cite{SingularIntegrals}

不失一般性, 在以下的计算中我们设 $r=0$, $\sigma(X_t)=constant=\sigma>0$ 回报函数 $f(x)=(x-K)^{+}, K>0$.  

由 Chebyshev 不等式, 我们有 
$$\mathbb{P}(M_T^n\geq \gamma)\leq \frac{E[(M_T^n)^p]}{\gamma^p}, \forall p>0$$

再由 Burkholder-Davis-Gundy 不等式有: 
\begin{equation}
\mathbb{E}[(M_T^n)^p]\leq C(p)\mathbb{E}[\left \langle M^n \right \rangle_T^{\frac{p}{2}}]
\end{equation}
这里我们用已知最优的结果 $C(p)=(2\sqrt{p})^p$.

对方程(2.7), 关于 $x$ 微分,我们有
\begin{equation}
\frac{\partial^2 u}{\partial t \partial x}(t, x) +\frac{1}{2}\sigma^2x^2\frac{\partial^3}{\partial x^3}u(t,x)=
-\sigma^2x\frac{\partial^2u}{\partial x^2}(t, x)
\end{equation}
对$\frac{\partial u}{\partial x}(t, X_t)-\frac{\partial u}{\partial x}(\varphi(t), X_{\varphi(t)})$ ,从 $\varphi(t)$ 到 $t$ 使用 It$\hat{o}$ 公式。
结合方程 (3.6), 我们有
\begin{equation}
\begin{split}
&\frac{\partial u}{\partial x}(t, X_t)-\frac{\partial u}{\partial x}(\varphi(t), X_{\varphi(t)})\\
&=\int_{\varphi(t)}^t (\frac{\partial^2 u}{\partial x\partial t}(\theta, 
X_\theta)+\frac{1}{2}\frac{\partial^3 u}{\partial x^3}(\theta, X_\theta)\sigma^2X^2_\theta) d\theta
+ \int_{\varphi(t)}^t \frac{\partial^2 u}{\partial x^2}(\theta, X_\theta)\sigma X_\theta dW_\theta\\
&=  \int_{\varphi(t)}^t \frac{\partial^2 u}{\partial x^2}(\theta, X_\theta)\sigma X_\theta (dW_\theta-\sigma d\theta)\\
\end{split}
\end{equation}

所以
\begin{equation}
M_T^n=\int_0^T\int_{\varphi(t)}^t \frac{\partial^2 u}{\partial x^2}(\theta, X_\theta)\sigma X_\theta (dW_\theta-\sigma d\theta)\sigma X_t dW_t
\end{equation}
\begin{equation}
\left \langle M^n \right \rangle_T=\int_0^T(\int_{\varphi(t)}^t \frac{\partial^2 u}{\partial x^2}(\theta, X_\theta)\sigma X_\theta (dW_\theta-\sigma d\theta))^2\sigma^2 X_t^2 dt
\end{equation}

我们假设 $p\geq 2$. 由广义 Minkowshi 不等式, 我们有:
\begin{equation}
\begin{split} 
\mathbb{E}[\left \langle M^n \right \rangle_T^{\frac{p}{2}}]
&=\mathbb{E}[(\int_0^T(\int_{\varphi(t)}^t \frac{\partial^2 u}{\partial x^2}(\theta, X_\theta)\sigma X_\theta (dW_\theta-\sigma d\theta))^2\sigma^2 X_t^2 dt)^{\frac{p}{2}}]\\
&\leq (\int_0^T(\mathbb{E}[(\int_{\varphi(t)}^t \frac{\partial^2 u}{\partial x^2}(\theta, X_\theta)\sigma X_\theta (dW_\theta-\sigma d\theta))^p\sigma^p X_t^p] )^{\frac{2}{p}}dt)^{\frac{p}{2}}\\
\end{split}
\end{equation}
考虑一个满足下面条件的新的概率$\widetilde{\mathbb{Q}}$: 
\begin{equation}
\frac{d\widetilde{\mathbb{Q}}}{d\mathbb{Q}}=e^{\sigma W_t-\frac{\sigma^2t}{2}}=\frac{X_t}{X_0}
\end{equation}
则在此新的概率$\widetilde{\mathbb{Q}}$ 下,$\widetilde{W}_t=W_t-\sigma t$ 是一个布朗运动。

由Cauchy 不等式,在概率 $\widetilde{\mathbb{Q}}$ 下, 我们有:
\begin{equation}
\mathbb{E}[\left \langle M^n \right \rangle_T^{\frac{p}{2}}]\leq x_0
(\int_0^T(\mathbb{E}_{\widetilde{\mathbb{Q}}}[(\int_{\varphi(t)}^t \frac{\partial^2 u}{\partial x^2}(\theta, X_\theta)\sigma X_\theta d\widetilde{W}_\theta)^{2p}])^{\frac{1}{p}}(\mathbb{E}[\sigma^{2p} X_t^{2p-2}] )^{\frac{1}{p}}dt)^{\frac{p}{2}}
\end{equation}
再次使用Burkholder-Davis-Gundy 不等式和广义 Minkowski 不等式,我们有:
\begin{equation}
\begin{split}
\mathbb{E}[\left \langle M^n \right \rangle_T^{\frac{p}{2}}]&\leq x_0
(\int_0^T(C(2p)\mathbb{E}_{\widetilde{\mathbb{Q}}}[(\int_{\varphi(t)}^t (\frac{\partial^2 u}{\partial x^2}(\theta, X_\theta)\sigma X_\theta)^2 d\theta)^{p}])^{\frac{1}{p}}(\mathbb{E}[\sigma^{2p} X_t^{2p-2}] )^{\frac{1}{p}}dt)^{\frac{p}{2}}\\
&\leq x_0C(2p)^{\frac{1}{2}}(\int_0^T(\int_{\varphi(t)}^t(\mathbb{E}_{\widetilde{\mathbb{Q}}}[ (\frac{\partial^2 u}{\partial x^2}(\theta, X_\theta)\sigma X_\theta)^{2p}])^{\frac{1}{p}} d\theta)(\mathbb{E}[\sigma^{2p} X_t^{2p-2}] )^{\frac{1}{p}}dt)^{\frac{p}{2}}\\
\end{split}
\end{equation}
\begin{equation}
\begin{split}
\mathbb{E}[X_t^{2p-2}]&=X_0^{2p-2}\mathbb{E}[e^{(2\sigma W_t-\sigma^2t)(p-1)}]\\
&=X_0^{2p-2}e^{2t(p-1)^2\sigma^2-p(t-1)\sigma^2}\\
&\leq
X_0^{2p-2}e^{\sigma^2p}, \forall p\in [2, +\infty)\\
\end{split}
\end{equation}
定义$\widetilde{C}(p)$ 如下 
\begin{equation}
\widetilde{C}(p)=
X_0^{2p-2}e^{\sigma^2p}, \forall p\in  [2, +\infty)
\end{equation}
另一方面我们有:
\begin{equation}
\mathbb{E}_{\widetilde{\mathbb{Q}}}[ (\frac{\partial^2 u}{\partial x^2}(\theta, X_\theta)\sigma X_\theta)^{2p}]=\frac{1}{(T-\theta)^p}\mathbb{E}_{\widetilde{\mathbb{Q}}}[N(d_1(\theta))^{2p}]
\end{equation}
这里
\begin{equation}
N(x)=\frac{1}{\sqrt{2\pi}}e^{\frac{-x^2}{2}}, d_1(\theta)=\frac{1}{\sigma\sqrt{T-\theta}}(log(\frac{X_\theta}{K})+\frac{1}{2}(T-\theta))
\end{equation}
记 $d_2(\theta)=\frac{1}{\sigma\sqrt{T-\theta}}(log(\frac{X_\theta}{K})-\frac{1}{2}(T-\theta))$, 此定义将在后面用到.

由$X_t=x_0exp(\sigma \widetilde{W}_t+\frac{1}{2}\sigma^2t)$, 我们有:
\begin{equation}
\mathbb{E}_{\widetilde{\mathbb{Q}}}[ (\frac{\partial^2 u}{\partial x^2}(\theta, X_\theta)\sigma X_\theta)^{2p}]
=(\frac{1}{2\pi(T-\theta)})^p\mathbb{E}_{\widetilde{\mathbb{Q}}}[exp(-p(\frac{log(\frac{x_0}{K})+\frac{1}{2}\sigma^2T+\sigma\widetilde{W}_{\theta}}{\sigma\sqrt{T-\theta}})^2)]
\end{equation}
定义常数$\lambda$ 如下: $\lambda=log(\frac{x_0}{K})+\frac{1}{2}\sigma^2T$, 则不等式为:
\begin{equation}
\begin{split}
\mathbb{E}_{\widetilde{\mathbb{Q}}}[ (\frac{\partial^2 u}{\partial x^2}(\theta, X_\theta)\sigma X_\theta)^{2p}]
&=(\frac{1}{2\pi(T-\theta)})^p\int^{+\infty}_{-\infty}\frac{1}{\sqrt{2\pi\theta}}e^{-\frac{z^2}{2\theta}}e^{-\frac{p}{T-\theta}(z+\frac{\lambda}{\sigma})^2}dz\\
&=(\frac{1}{2\pi(T-\theta)})^p\sqrt{\frac{T-\theta}{T-\theta+2p\theta}}e^{-\frac{p}{T-\theta+2p\theta}\frac{\lambda^2}{\sigma^2}}\\
&\leq \frac{\sqrt{T-\theta}}{(2\pi(T-\theta))^p}\frac{\sigma}{\lambda\sqrt{2p}}e^{-\frac{1}{2}}\\
\end{split}
\end{equation}
所以 
\begin{equation}
\begin{split}
\mathbb{E}[\left \langle M^n \right \rangle_T^{\frac{p}{2}}]&\leq
 x_0C(2p)^{\frac{1}{2}}(\int_0^T(\int_{\varphi(t)}^t (T-\theta)^{\frac{1}{2p}-1} d\theta)\frac{1}{2\pi}(\frac{\sigma}{\sqrt{2p}\lambda}e^{-\frac{1}{2}})^{\frac{1}{p}}x_0^{2-\frac{2}{p}}e^{\sigma^2}dt)^{\frac{p}{2}}\\
 &=x_0^pC(2p)^{\frac{1}{2}}(\frac{1}{2\pi})^{\frac{p}{2}}(\frac{\sigma}{\lambda\sqrt{2\pi}})^{\frac{1}{2}}e^{\frac{2\sigma^2p-1}{4}}
 (\int_0^T\int_{\varphi(t)}^t (T-\theta)^{\frac{1}{2p}-1} d\theta dt)^{\frac{p}{2}}\\
 \end{split}
 \end{equation}
 \begin{equation}
 \begin{split}
 (\int_0^T\int_{\varphi(t)}^t (T-\theta)^{\frac{1}{2p}-1} d\theta dt)^{\frac{p}{2}}
 &=(2p\int_0^T((T-\varphi(t))^{\frac{1}{2p}}-(T-t)^{\frac{1}{2p}}) dt)^{\frac{p}{2}}\\
 &=(2p\frac{T}{n}\sum_{i=0}^{n-1}(T-t_i)^{\frac{1}{2p}}-\int_0^T(T-t)^{\frac{1}{2p}} dt)^{\frac{p}{2}}\\
 &=(2p)^{\frac{p}{2}}(\frac{T}{n})^{\frac{2p+1}{4}}(\sum_{i=1}^{n}i^{\frac{1}{2p}}-\int_0^n(n-v)^{\frac{1}{2p}} dv)^{\frac{p}{2}}\\
 &=(2p)^{\frac{p}{2}}(\frac{T}{n})^{\frac{2p+1}{4}}(\sum_{i=1}^{n}i^{\frac{1}{2p}}-\frac{2p}{2p+1}n^{\frac{2p+1}{2p}})^{\frac{p}{2}}\\
 \end{split}
 \end{equation}
 由于$0<\frac{1}{2p}=q<1$, 我们知道函数 $g(x)=x^q$ 是凹函数(concave). 由$g(x)$的凹性质,我们有:
 \begin{equation}
 \sum_{i=1}^{n-1}i^q+\sum_{i=1}^{n-1}\frac{(i+1)^q-i^q}{2}<\int_1^{n}x^qdx
 \end{equation}
 所以
 \begin{equation}
 \begin{split}
 \sum_{i=1}^{n-1}i^q
 &<\frac{1}{q+1}(n^{q+1}-1)-\frac{1}{2}(n^q-1)\\
 &<\frac{n^{q+1}}{q+1}-\frac{n^q}{2}\\
 \end{split}
 \end{equation}
 \begin{equation}
 \sum_{i=1}^{n}i^q=\sum_{i=1}^{n-1}i^q+n^q<\frac{n^{q+1}}{q+1}+\frac{n^q}{2}
 \end{equation}

 \begin{equation}
 \sum_{i=1}^{n}i^{\frac{1}{2p}}-\frac{2p}{2p+1}n^{\frac{2p+1}{2p}}<\frac{n^{\frac{1}{2p}}}{2}
 \end{equation}
 因此:
 \begin{equation}
 \begin{split}
 \mathbb{E}[\left \langle M^n \right \rangle_T^{\frac{p}{2}}]
 &\leq x_0^pC(2p)^{\frac{1}{2}}(\frac{1}{2\pi})^{\frac{p}{2}}(\frac{\sigma}{\lambda\sqrt{2\pi}})^{\frac{1}{2}}e^{\frac{2\sigma^2p-1}{4}}
 (2p)^{\frac{p}{2}}(\frac{T}{n})^{\frac{2p+1}{4}}(\frac{1}{2})^{\frac{p}{2}}n^{\frac{1}{4}}\\
 \end{split}
 \end{equation}
 故而对于 $p\geq 2$,
 \begin{equation}
 \mathbb{P}(M_T^n\geq \gamma)\leq L(p)=C_0\frac{1}{\gamma^p}C(p)x_0^pC(2p)^{\frac{1}{2}}(\frac{1}{2\pi})^{\frac{p}{2}}e^{\frac{\sigma^2p}{2}}
 p^{\frac{p}{2}}(\frac{T}{n})^{\frac{p}{2}}
 \end{equation}
 这里
 $C_0=T^{\frac{1}{4}}(\frac{\sigma}{\lambda\sqrt{2\pi}})^{\frac{1}{2}}e^{-\frac{1}{4}}$ 是一个与$p, n$ 无关的常数。  
 通过令等式 $\frac{\partial \log(L(p))}{\partial p}=0$ 成立。 我们找到最优的 $p$。
 \begin{equation}
 \log(L(p))=\frac{3}{2}p\log(p)+\log(C_0)-\frac{p}{2}\log(C_1n)
 \end{equation}
 这里 $C_1=\frac{\gamma^2\pi e^{-\sigma^2}}{16x_0^2T}$ 是与 $p, n$ 无关的常数。
 \begin{equation}
 0=\frac{\partial\log(L(p))}{\partial p}=\frac{1}{2}(3+3\log(p)-\log(C_1n)) 
 \end{equation}
 \begin{equation}
 p=\frac{(C_1n)^{\frac{1}{3}}}{e}
 \end{equation}
 最优的$p$ 是 $p*$:
 \begin{equation}
 p*=
 \begin{cases}
 2, \quad  \log(C_1n)<3\log(2)+3\\
 \frac{(C_1n)^{\frac{1}{3}}}{e}, \quad \log(C_1n)\geq 3\log(2)+3\\
 \end{cases}
 \end{equation}
 因此 $L(p)$ 的最小值是:
 \begin{equation}
 L(p*)=
 \begin{cases}
 \frac{8C_0}{C_1n}, \quad \log(C_1n)<3\log(2)+3\\
 C_0e^{-\frac{3(C_1n)^{\frac{1}{3}}}{2e}}, \quad \log(C_1n)\geq 3\log(2)+3\\
 \end{cases}
 \end{equation}
 现在我们得到了一个 $N$ 的估界:
 \textbf{Proposition}
 \begin{equation}
 \mathbb{P}(M_T^n\geq \gamma)\leq 
 \begin{cases}
 \frac{128\sigma^{\frac{1}{2}}x_0^2T^{\frac{5}{4}}e^{\sigma^2}}{e^{\frac{1}{4}}\gamma^2\pi (\log(\frac{x_0}{K})+\frac{1}{2}\sigma^2T)^{\frac{1}{2}}(2\pi)^{\frac{1}{4}}}\frac{1}{n}, \quad \log(C_1n)<3\log(2)+3\\
 T^{\frac{1}{4}}(\frac{\sigma}{(\log(\frac{x_0}{K})+\frac{1}{2}\sigma^2T)\sqrt{2\pi}})^{\frac{1}{2}}e^{-\frac{1}{4}}e^{-\frac{3\gamma^{\frac{2}{3}}\pi^{\frac{1}{3}}}{2ex_0^{\frac{2}{3}}T^{\frac{1}{3}}16^{\frac{1}{3}}e^{\frac{\sigma^2}{3}}}n^{\frac{1}{3}}}, \quad \log(C_1n)\geq 3\log(2)+3\\
 \end{cases}
 \end{equation}


% section 参数N (end)

\section{参数M} % (fold)
\label{sec:M}
在确定了 $N$ 的迭代初始值之后 $N_0$,我们可以使用该值来确定一个合适大小的 $M$ 的迭代初始值。
一个合适的 $M$ 的初始值应该满足如下条件:在 $N$ 为其初始迭代值时, $M$ 的值应该使得模特卡罗模拟出的概率 $\mathbb{P}(M_T^N<\epsilon)$ 收敛稳定。
也即,固定$N$ 为其初始值不变,当 $M$ 在其迭代初始值附近扰动时,相应的 $\mathbb{P}(M_T^N<\epsilon)$ 也只是微小扰动。动态调优的技术可以被应用在这里。

在我们的问题背景下,通过动态调优和经验发现,一个较为恰当的 $M$ 初始值可以为 $M=10^6$, 我们通过比较 $M=10^6+i*100$, $i=0,...,10$ 
所对应的 $\mathbb{P}(M_T^N<\epsilon)$来调整 $M$, 若对应的概率在 $10^{-3}$范围内,则将$M$ 减小为原先的一半,否则增大 $M$ 为原先的两倍。
重复此过程,知道发现满足条件的最小的 $M$。 我们然后固定这个最优的 $M$, 然后尝试找出最优的 $N$。

根据前面的理论,最优的 $N$ 存在于 $[0, N_0]$ 区间内。我们仍然使用二分法寻找最优的$N$, 首先我们令 $N_1=N_0/2$, 在此参数下模拟相应的
$\mathbb{P}(M_T^N<\epsilon)$, 若此模拟概率大于 $Prob$, 则令 $N_2=N_1/2$, 反之,则令 $N_2=\frac{N_0+N_1}{2}$。
重复此过程直至找到最优的$N$。

% section 参数M (end)




