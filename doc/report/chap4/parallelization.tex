\chapter{并行化算法}
\label{chp:4}

算法并行化的第一步通常是设计一个符合待处理问题及处理器架构特点的并行方案,然后根据该方案确定相应的编程模型,最后分析并行程序的性能,有针对性地做具体的优化。
并行程序的性能优化和普通串行程序有许多共通之处。具体来说主要分为两个大类:针对内存使用的优化和针对for循环的改进。
由于日益增长的处理器速度和发展相对较缓的内存速度之间逐渐拉开了差距,内存通常成为程序蓄能的瓶颈。现代处理器通常采用分层内存体系,高带宽,低延迟(lower latency),低能耗的高速缓存对提升数据局部性(data locality)至关重要。针对缓存优化的方法有很多,例如对齐数据(data alignment)使得缓存的加载更有效率,依据缓存的大小处理相应的数据块(cache blocking),预读取(prefetching)提前将需要处理的数据加载到缓存中,或者通过流式存储技术(streaming store)将非时间相关(nontemporal)的计算结果"绕过"缓存直接写入主内存从而避免"污染"缓存其他数据等。和数据代码缓存一样,页表缓存(Translation Lookaside Buffer)也是影响性能的一个因素。页表缓存存储了一部分标签页表条目,用于改进从虚拟内存地址到物理地址的转译速度,对于特定问题,通过使用不同尺寸的页面大小,重构数据可以避免页表缓存中的热点。
针对for循环优化的一般技术包括循环展开(loop unrolling),分块循环(loop tiling), 循环互换(loop interchange),循环合并(loop fusion),循环偏移(loop skewing),循环剥离(loop peeling)等。这些改进很多时候也是为了更好地使用内存或者为进一步并行化做准备。
目前绝对意义上的串行程序在实际中已不多见,因为在编译过程中编译器会或多或少引入不同程度的并行化处理。然而由于编译器优化的前提是基于代码的正确性,串行程序的编写和调试难度远低于并行程序。本章将详细介绍。。


通过充分使用高速缓存以提高数据的局部性(data locality),

包括locality, alignment, prefetching, streaming store 

对于多次compute intensive的优化
充分考虑处理器片上高速存储单元和通信带宽,
