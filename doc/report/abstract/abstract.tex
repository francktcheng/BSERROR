\begin{abstract}
本文研究了基于Black-Scholes模型的离散时间版本的Delta对冲策略与理论上的Delta对冲策略之间的误差,
该误差随着离散时间点的抽样次数的增加而减小。由于实际交易中交易成本一类的问题,所以在一定的误差风险范围内,
我们希望能够用最少的交易次数来完成既定的交易策略。
给定抽样次数,我们使用蒙特卡洛方法计算出该误差在某一可接受范围内的概率。若该概率能够高于一个事先确定可接受值,则认为这个抽样次数是可接受的。
本文利用高性能计算的相关技术针对Intel平台实现了该应用的并行化。通过详细剖析该应用中的并行度,解除数据的依赖性,我们首先针对单机利用多线程
和矢量化提出了两种并行方案。为了将应用扩展至更大的平台,我们使用Master-Slave模型利用消息传递模式通过MPI实现了多机上运行的并行版本。
最终目的是为了搜索最优离散时间点的抽样次数。
我们开创性地通过理论给出最优抽样次数的数学上界,然后通过二分法在由该上界给出的闭区间内搜索最优解。初步的测试表明并行程序相对于串行版本获得了极大的性能提升。并行程序在MIC集群上也获得了良好的扩展性。




\end{abstract}
