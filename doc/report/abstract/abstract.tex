\begin{abstract}
本文研究了基于Black-Scholes模型的离散时间版本的Delta对冲策略与理论上的Delta对冲策略之间的误差,
该误差随着离散时间点的抽样次数的增加而减小。由于实际交易中交易成本一类的问题,所以在一定的误差风险范围内,
我们希望能够用最少的交易次数来完成既定的交易策略。
给定抽样次数,我们使用蒙特卡洛方法计算出该误差在某一可接受范围内的概率。若该概率能够高于一个事先确定可接受值,则我们认为这个抽样次数是可接受的。
本文利用高性能计算在Intel最新的MIC架构上计算出了最优的离散时间点的抽样次数。
我们首先通过理论给出最优抽样次数的上界,然后通过二分法在由该上界给出的闭区间内搜索最优解。
由于本文所讨论的数据依赖性本质上源自于独立高斯分布的随机数组成的数组,
我们对此给出了两种并行思路。
在我们的第一种并行思路中,我们只需要生成一条随机流,每次计算某段子集的时候空闲的线程利用随机数发生器生成下一个紧邻子集对应的随机数。
所有随机数由全部线程共享,从而解除了数据的依赖性。 在第二种并行思路中,每个线程保 有一条私有的随机数流,根据自己计算的进度按需生成相应的随机数。
这种情况下对于数据依赖性的处理则在于随机数的“种子”。


\end{abstract}
