\documentclass[12pts,a4paper]{report}

\usepackage{fontspec}
%\setromanfont{LiHei Pro} % 儷黑Pro
\setromanfont{Hiragino Sans GB} % 儷黑pro
\setmonofont[Scale=0.8]{Courier New} % 等寬字型
\XeTeXlinebreaklocale "zh"
\XeTeXlinebreakskip = 0pt plus 1pt

\renewcommand\abstractname{摘要}
\renewcommand\appendixname{附录}
\renewcommand\bibname{参考文献}
\renewcommand\chaptername{章节}
%\renewcommand\contentsname{}
%\renewcommand\indexname{}
%\renewcommand\listfigurename{}
%\renewcommand\listtablename{}
%\renewcommand\partname{}
%\renewcommand\refname{}

\usepackage[thmmarks]{ntheorem}
% 定理类环境宏包,其中 amsmath 选项
% 用来兼容 AMS LaTeX 的宏包
 
 %=== 配合上面的ntheorem宏包产生各种定理结构,重定义一些
 %正文相关标题 ===
 %\theoremstyle{plain}
 %\theoremheaderfont{\normalfont\rmfamily\CJKfamily{hei}}
 %\theorembodyfont{\normalfont\rm\CJKfamily{kai}} \theoremindent0em
 %\theoremseparator{\hspace{1em}} \theoremnumbering{arabic}
 %\theoremsymbol{}          %定理结束时自动添加的标志
 \newtheorem{definition}{\hspace{2em}定义}[section]
 %\newtheorem{definition}{\hei 定义}[section]
 %!!!注意当section为中国数字时,[section]不可用!
 \newtheorem{proposition}{\hspace{2em}命题}[section]
 \newtheorem{property}{\hspace{2em}性质}[section]
 \newtheorem{lemma}{\hspace{2em}引理}[section]
 %\newtheorem{lemma}[definition]{引理}
 \newtheorem{theorem}{\hspace{2em}定理}[section]
 \newtheorem{axiom}{\hspace{2em}公理}[section]
 \newtheorem{corollary}{\hspace{2em}推论}[section]
 \newtheorem{exercise}{\hspace{2em}习题}[section]
 \theoremsymbol{$\blacksquare$}
 \newtheorem{example}{\hspace{2em}例}[section]
 \theoremstyle{nonumberplain}
 %\theoremheaderfont{\CJKfamily{hei}\rmfamily}
 %\theorembodyfont{\normalfont \rm \CJKfamily{song}} \theoremindent0em
 \theoremseparator{\hspace{1em}} \theoremsymbol{$\blacksquare$}
 \newtheorem{proof}{\hspace{2em}证明}
  
\usepackage{bbm}%可以使用空心数字
\usepackage{extarrows}%long arrow



%\usepackage{amsmath, amssymb, amsthm} % AMS packages
%\usepackage{amssymb}
%\usepackage{amsthm}
\usepackage{inputenc}
\usepackage{verbatim}
\usepackage[dvips]{lscape,graphicx}
\usepackage{pstricks} %graphics objects
\usepackage{pst-node}
\usepackage{float}
\usepackage{subfigure}
\usepackage{amsmath}
\usepackage{amsfonts}
\usepackage{fullpage}
\usepackage{vmargin}
\usepackage[normalem]{ulem}           
\usepackage{array}
\usepackage{multirow} 
\usepackage{moreverb}
\usepackage{epsfig}
\usepackage{subfigure}
\usepackage{nopageno}
\usepackage{url}                       
\usepackage{hyperref}                 
\usepackage{color}                     
\usepackage{pifont}                     
\usepackage{fancybox}
\usepackage{listings}                           
\usepackage{bibunits}
%\usepackage{bibtopic}
\usepackage{graphicx}
%\usepackage{ams}

% \usepackage {algorithm}
% \usepackage{algorithmic} % AMS packages
\usepackage{algorithm}
\usepackage{algorithmicx}
\usepackage{algpseudocode}

\usepackage{fancyhdr}
\setlength{\headheight}{12pt}
\fancyhf{}
\pagestyle{fancy}
\renewcommand{\chaptermark}[1]{\markboth{#1}{}}
\renewcommand{\sectionmark}[1]{\markright{\thesection\ #1}}
\fancyhf{} \fancyhead[LE,RO]{\bfseries\thepage}
\fancyhead[LO]{\bfseries\rightmark}
\fancyhead[RE]{\bfseries\leftmark}
\renewcommand{\headrulewidth}{0.5pt}

\addtolength{\headheight}{0.5pt}
\renewcommand{\footrulewidth}{0pt}
\fancypagestyle{plain}{ \fancyhead{}
\renewcommand{\headrulewidth}{0pt}} 


% commandes persos pour tabuler les blocs
% dans un algorithme

\newcommand{\ztab}[1]{\noindent
#1\\
}

\newcommand{\tab}[1]{\noindent
\begin{tabular}{@{} l|l @{}}
&
#1\\
\end{tabular}\\
}

\newcommand{\ttab}[1]{\noindent
\begin{tabular}{@{} l|l|l @{}}
&&
#1\\
\end{tabular}\\
}

\newcommand{\tttab}[1]{\noindent
\begin{tabular}{@{} l|l|l|l @{}}
&&&
#1\\
\end{tabular}\\
}

\newcommand{\flga}[0]{\leftarrow}
\newcommand{\fldr}[0]{\rightarrow}
\newcommand{\non}[0]{\invneg}          % non logique

\newcommand{\paren}[1]{\left(#1\right)} % entre parentheses
\newcommand{\croch}[1]{\left[#1\right]} % entre crochets

\newcommand{\brut}[1]{\textrm{#1}}  % texte normal
\newcommand{\gras}[1]{\textbf{#1}}  % texte en gras
\newcommand{\sase}[1]{\textsf{#1}}  % texte sans empattement (SAns SErif)


\floatstyle{ruled}
\newfloat{algorithm}{thp}{lop}
\floatname{algorithm}{Algorithm}

%\floatname{algorithm}{Procedure}
%\renewcommand{\algorithmicrequire}{\textbf{输入参数}}
%\renewcommand{\algorithmicensure}{\textbf{输出结果}}
\renewcommand{\algorithmicrequire}{输入参数}
\renewcommand{\algorithmicensure}{输出结果}


